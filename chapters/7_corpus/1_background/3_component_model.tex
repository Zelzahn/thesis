\section{The Component Model}
\label{chap:component_model}

% [Wat is het component model en waarom willen we het. Ook vertellen dat je het kan zien als een guest/host-architectuur, maar dat het perfect mogelijk is om meerdere componenten samen te weven.]

The WebAssembly Component Model is an architecture for building interoperable \gls{Wasm} liraries, applications and environments. These components can be seen as a wrapper around a core module, or other components, which express their interfaces and dependencies via \gls{WIT} and the canonical \gls{ABI}. Unlike core modules, components may not export \gls{Wasm} memory, reinforcing \gls{Wasm} sandboxing and facilitating interoperation between languages with different memory assumptions.

An \gls{ABI} can be seen as an agreement on how to pass around data in a binary format, specifically concerned with the data layout at the bits-and-bytes level. The Canonical \gls{ABI} defined by the component model, specifies how the \gls{WIT} type definitions are translated to bits and bytes. Internally, a C and a Rust component might represent strings in a quite different way, but the canonical \gls{ABI} provides a format for them to pass strings across the boundary between them.

In regard to \gls{WASI}, the component model is the staple feature of its second preview, but it is possible to make use of the \gls{WASI} interfaces without the component model, and thus this model is entirely optional. By way of comparison to a traditional \gls{OS}, the Component Model fills the role of an \gls{OS}'s process model, defining how processes start up and communicate with each other, while \gls{WASI} fills the role of an \gls{OS}'s many I/O interfaces.

To compose multiple components together \texttt{wasm-tools}~\cite{gh:wt} can be used, or visually using the builder~\cite{builder} app. Specifically for Rust, \texttt{cargo-component}~\cite{gh:cc} is also an option.

\subsection{Toolchain}
\label{sec:guest}
% [Op welke manieren kan je allemaal code schrijven dat dan naar een preview 2 component gecompileerd kan worden. Verschillende tooling over programmeertalen heen de specifieke voor Rust. Alsook hoe we van preview1 component naar preview2 kunnen gaan.]

In Rust, \texttt{cargo-component} can be used to compile code to a preview 2 component. In essence, compiling to \texttt{Preview 2} means compiling to \texttt{wasm32-wasi} and then converting it to a component via an adapater and the \texttt{wasm-tools component new} subcommand. This component then adheres to the \gls{WIT} interface specified in the configuration file. The adaption is needed because there's no first-class support for \texttt{Preview 2} yet. Mainstream support for this is planned for early 2025~\cite{rust:p2}.

Under the hood, \texttt{cargo-component} relies upon \texttt{wit-bindgen}~\cite{gh:wit-bindgen} for binding with the interface. Besides Rust, \texttt{wit-bindgen} also supports the following languages: C, Java, Go and C\#. For JavaScript, ComponentizeJS~\cite{gh:cjs} can be used.

\subsubsection{Adapter modules}

The Wasmtime runtime publishes adapter modules with each release, they provide the bridge between the \texttt{Preview 1} \gls{ABI} and the \texttt{Preview 2} \gls{ABI}. The following three modules are provided:

\begin{itemize}
    \item Command: For command-line applications
    \item Reactor: Applications that don't have a \texttt{main} function
    \item Proxy: For applications fed into \texttt{wasmtime serve}
\end{itemize}

The \texttt{wasmtime serve} subcommand runs a component inside the \texttt{wasi:http/proxy} world, supporting the sending and receiving of HTTP requests.

\subsection{Running components}
\label{sec:host}
% [Uitlegen dat je runtimes hebt en dat je rechtstreeks daarop kan draaien of dat er ook visence, compiling to preview 2 means compiling to \gls{Wasm} and then converting it to a component via an adapater.

Running a component is done by calling one of its exports. This can require a custom host, otherwise the \texttt{wasmtime} command line can be used.

The job of a custom host is to load a component and execute it through the usage of a \gls{Wasm} runtime. See section~\ref{sec:runtimes} for a shortlist of the available ones. To guarantee a correct execution, it is important to make sure that any missing interface imports are filled in here, see the earlier section~\ref{sec:wit}. When using \texttt{wit-bindgen}, this is done via the \texttt{with} option inside the \texttt{bindgen}~\cite{crates:bindgen} macro.

When the component exports the \path{wasi:cli/run} interface, and imports only interfaces listed in the \path{wasi:cli/command} world, it is considered a command component. Command components can be executed by the \texttt{wasmtime run} subcommand. This will compile the module to native code, instantiate it and optionally execute an export.

